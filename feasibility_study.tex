\chapter{Feasibility study}

The design of the module for image super resolution is quite challenging, considering that various kinds of technical aspects involved in this project. The project has a multitude of problems that it will have to deal with. The main challenge is the issue that the system has to have a generic system that will be able to deal with the inputs that the system gets.

Having a good sample training set is really vital for such kind of an approach so as to obtain the maximum accuracy in the output. Since this is not so much of an issue as we can easily create large amount of data set from a lot of good quality images. The main issue with the dataset is having to deal with the diversity of the images as we need to have a wide range of data point to make it working as a general system.

In this Project, we demonstrate how image processing, and machine learning can be used to enhance the clarity of an image. To that end, we propose two segments: one is an image processing part, and the other is a neural network modal that will work on that image and fix image issues.

The proposed system will follow the below given processing model to enhance image.

\begin{itemize}
    \item The image will can be uploaded from the device, both computer or a mobile
    \item The obtained image is then sent to a sever
    \item A small amount of preprocessing is done on the image
    \item The images if processed using the image super resolution model
    \item Result is sent back to the device
    \item The image that has been enhanced in the server is made available to the user via the UI
\end{itemize}
