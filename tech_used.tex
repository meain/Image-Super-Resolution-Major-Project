\chapter{Technology used}

\section{Image super resolution model}

\subsection{Tensorflow}
TensorFlow™ is an open source software library for high performance numerical computation. Its flexible architecture allows easy deployment of computation across a variety of platforms (CPUs, GPUs, TPUs), and from desktops to clusters of servers to mobile and edge devices. Originally developed by researchers and engineers from the Google Brain team within Google’s AI organization, it comes with strong support for machine learning and deep learning and the flexible numerical computation core is used across many other scientific domains

\subsection{Keras}
Keras is an open source neural network library written in Python. It is capable of running on top of TensorFlow, Microsoft Cognitive Toolkit, Theano, or MXNet. Designed to enable fast experimentation with deep neural networks, it focuses on being user-friendly, modular, and extensible. It was developed as part of the research effort of project ONEIROS (Open-ended Neuro-Electronic Intelligent Robot Operating System.

\section{Frontend}

\subsection{React}
React is a front-end library developed by Facebook. It is used for handling the view layer for web and mobile apps. ReactJS allows us to create reusable UI components. It is currently one of the most popular JavaScript libraries and has a strong foundation and large community behind it.

\section{Backend}

\subsection{Flask}
Flask is a microframework for Python based on Werkzeug and using Jinja2 for templating. It helps to easily prototype and iterate on various version of the backend api that is being provided.

\subsection{Gunicorn}
Gunicorn 'Green Unicorn' is a Python WSGI HTTP Server for UNIX. It's a pre-fork worker model. The Gunicorn server is broadly compatible with various web frameworks, simply implemented, light on server resources, and fairly speedy.
