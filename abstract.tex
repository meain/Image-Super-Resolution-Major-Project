\centerline{\large{\bfseries{ABSTRACT}}}

\hspace{1in}

\normalsize

In most digital imaging applications, high resolution images or videos are usually desired for later image processing and analysis. The desire for high image resolution stems from two principal application areas: improvement of pictorial information for human interpretation; and helping representation for automatic machine perception. Image resolution describes the details contained in an image, the higher the resolution, the more image details.
Super-resolution (SR) are techniques that construct high-resolution (HR) images from several observed low-resolution (LR) images, thereby increasing the high frequency components and removing the degradation caused by the imaging process of the low resolution camera. The basic idea behind SR is to combine the non-redundant information contained in multiple low- resolution frames to generate a high-resolution image. A closely related technique with SR is the single image interpolation approach, which can be also used to increase the image size. However, since there is no additional information provided, the quality of the single image interpolation is very much limited due to the ill-posed nature of the problem, and the lost frequency components cannot be recovered. In the SR setting, however, multiple low-resolution observations
are available for reconstruction, making the problem better constrained. The non-redundant information contained in the these LR images is typically introduced by subpixel shifts between them. These subpixel shifts may occur due to uncontrolled motions between the imaging system and scene, e.g., movements of objects, or due to controlled motions, e.g., the satellite imaging system orbits the earth with predefined speed and path.
In our project we attempt to use deep convolutional neural nets to train a model to output enlarged images given a picture. We also attempt to exploit generative-adverserial nets in favour of this project.